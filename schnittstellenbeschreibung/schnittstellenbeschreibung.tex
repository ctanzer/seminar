\documentclass{article}

\usepackage{listings}
\usepackage{hyperref}
\usepackage{url}

\setlength{\parindent}{0pt}

\begin{document}
	\section*{PBAS}
	Verwendung:
	\begin{lstlisting}
	python pbas_multiprocessing.py "TOPICNAME"
	\end{lstlisting}
	\vspace{1em}
	Eingabe: \\
	
	String mit Namen des Eingabetopics. Dieses muss vom Typ sensor\_msgs.msg.Image sein.\\
	\url{http://docs.ros.org/kinetic/api/sensor_msgs/html/msg/Image.html}
	\vspace{1em}
	\vspace{1em}
	
	Ausgabe: \\
	
	Topic "pbas\_segmentation" vom Typ sensor\_msgs.msg.Image.\\
	\url{http://docs.ros.org/kinetic/api/sensor_msgs/html/msg/Image.html}
	
	\section*{SuBSENSE}
	Verwendung:
	\begin{lstlisting}
	python subsense_multiprocessing.py "TOPICNAME"
	\end{lstlisting}
	\vspace{1em}
	Eingabe: \\
	
	String mit Namen des Eingabetopics. Dieses muss vom Typ sensor\_msgs.msg.Image sein.\\
	\url{http://docs.ros.org/kinetic/api/sensor_msgs/html/msg/Image.html}
	\vspace{1em}
	\vspace{1em}
	
	Ausgabe: \\
	
	Topic "subsense\_segmentation" vom Typ sensor\_msgs.msg.Image.\\
	\url{http://docs.ros.org/kinetic/api/sensor_msgs/html/msg/Image.html}
\end{document}